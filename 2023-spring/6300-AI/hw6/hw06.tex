\documentclass[12pt]{article}

\usepackage{times}
\usepackage{graphicx}
\usepackage{url}
\usepackage{amsmath}

\setlength{\textwidth}{6.5in}
\setlength{\textheight}{8.9in}
\setlength{\oddsidemargin}{0.0in}
\setlength{\topmargin}{0.05in}
\setlength{\headheight}{-0.05in}
\setlength{\headsep}{0.0in}

\begin{document}

\begin{center} 
{\bf CS 6300} \hfill {\large\bf HW06: Q Learning with
    Functional Approximation} \hfill {\bf Due March 13, 2023}
\end{center}

\noindent
Please use the \LaTeX\ template to produce your writeups. See the
Homework Assignments page on the class website for details.  Hand in
through gradescope.


\section{Functional Approximation}

We revisit the simplied version of blackjack from Homework 5. The deck is infinite and the
dealer always has a fixed count of 15.  The deck contains cards 2
through 10, J, Q, K, and A, each of which is equally likely to appear
when a card is drawn.  Each number card is worth the number of points
shown on it, the cards J, Q, and K are worth 10 points, and A is worth
11.  At each turn, you have two possible actions: either {\it hit} or {\it stay}.

Unhappy with your experience with basic Q-learning, you decide to
featurize your Q-values.  Consider the two feature functions:

$$
f_1(s,a) = \left\{ \begin{array}{ll}
                  0  & a=stay \\
                  +1 & a=hit, s \geq 15 \\
                 -1  & a=hit, s < 15
                 \end{array} \right.
\quad \hbox{and} \quad
f_2(s,a) = \left\{ \begin{array}{ll}
                  0  & a=stay \\
                  +1 & a=hit, s \geq 18 \\
                 -1  & a=hit, s < 18
                 \end{array} \right.
$$

\noindent
Which of the following partial policy tables may be represented by the
featurized Q-values unambiguously (without ties)?  Derive your answers
for each policy table.

\begin{center}
\begin{tabular}{ccccc}
\begin{tabular}{|c|c|} \hline
$s$ & $\pi(s)$ \\ \hline
14  & hit \\
15  & hit \\
16  & hit \\
17  & hit \\
18  & hit \\
19  & hit \\ \hline
\end{tabular} &
\begin{tabular}{|c|c|} \hline
$s$ & $\pi(s)$ \\ \hline
14  & stay \\
15  & hit \\
16  & hit \\
17  & hit \\
18  & stay \\
19  & stay \\ \hline
\end{tabular} &
\begin{tabular}{|c|c|} \hline
$s$ & $\pi(s)$ \\ \hline
14  & hit \\
15  & hit \\
16  & hit \\
17  & hit \\
18  & stay \\
19  & stay \\ \hline
\end{tabular} &
\begin{tabular}{|c|c|} \hline
$s$ & $\pi(s)$ \\ \hline
14  & hit \\
15  & hit \\
16  & hit \\
17  & hit \\
18  & hit \\
19  & stay \\ \hline
\end{tabular} &
\begin{tabular}{|c|c|} \hline
$s$ & $\pi(s)$ \\ \hline
14  & hit \\
15  & hit \\
16  & hit \\
17  & stay \\
18  & hit \\
19  & stay \\ \hline
\end{tabular} \\
(a) & (b) & (c) & (d) & (e)
\end{tabular}
\end{center}

\noindent \textbf{Answer}

\noindent Given the features above, we have the following equation for this functional approximation, $Q(s,a) = w_1f_1 + w_2f_2$.\\
Since $f_1$ and $f_2$ are 0 when $a = stay$, we know $Q(s, stay) = 0$. \\
There are now 3 other possible states when $a = hit$:

$$
Q(s,hit) = \left\{ \begin{array}{ll}
                  -w_1 - w_2  & s < 15 \\
                  w_1 - w_2 & 15 \leq s < 18 \\
                 w_1 + w_2  & 18 \leq s
                 \end{array} \right.
$$

Given this piecewise equation, we can deduce what the possible policies might be. We don't know what $w_1$ and $w_2$ are, but we can investigate their relationship to find which patterns are possible.

The equation that I've derived above tells us that the policy for $s < 15$ (14) will all be the same, $15 \leq s < 18$ (15, 16, 17) will all be the same, and $18 \leq s$ (18, 19) will all be the same. This means that policy d and policy e are not possible since the action for 18 and 19 are different.

Now we need to investigate policies a, b, and c. Examining the piecewise equation, we can see that the Q values for $s < 15$ should be opposite to $18 \leq s$. That's because $-w_1 - w_2 = -1(w_1 + w_2)$, meaning the Q values will have opposite signs and the policy action will be different. That rules out policies a and b, since in those policies, the action for 14 is the same  as the action for 18 and 19.

This leaves c as the only possible policy. Here's an example of weights that would make policy c possible: $w_1 = 1, w_2 = 2$. In this case:

$$
Q(s,hit) = \left\{ \begin{array}{ll}
                  -3  & s < 15 \\
                  -1 & 15 \leq s < 18 \\
                 3  & 18 \leq s
                 \end{array} \right.
$$

This works if a negative Q value maps to hit and a positive to stay. If not, we can just flip the signs on the both weights to make it work.

Thus policy c is the only possible partial policy table that may be represented by the featurized Q-values.
\end{document}

